\documentclass[UTF8,aspectratio=169]{beamer}%aspectratio=169将页面调整为16:9
\usepackage{graphicx} % Required for inserting images
\usepackage[utf8]{inputenc}
\usepackage{ctex}


%更改主题
\usetheme{Copenhagen}%主题选Copenhagen
\usecolortheme{beaver}%但是颜色选beaver
%关闭导航标志
\setbeamertemplate{navigation symbols}{}
%关闭目录导航
\setbeamertemplate{headline}{}
%item变隐藏形式
\setbeamercovered{transparent}

%打印设置
%\usepackage{pgfpages}
%\pgfpagesuselayout{4 on 1}[a4paper, border = 5mm]%一张纸打印四个

\title{Beamer}
\author{Rony Wang}
\date{December 2023}



\begin{document}

\maketitle

%目录
\begin{frame}{Title Page}
    \tableofcontents
\end{frame}

\section{Introduction}
\begin{frame}{Frame Title}
    \begin{itemize}
        \item<1-> Collaboration %第一张slide显示
        \item<2-> Version history 1 %第二张slide显示
        \item<3-> Version history 2 %第三张slide显示
    \end{itemize}
    %如果我想在点击之后出现,可以这样进行:
    \only<1>{这个文字只会出现在第一页.}
    \onslide<2>{This text only appear on the second slide.}
    \onslide<3>{This text will continue appear on the third slide.}
   

    首先,第一页\alert<1>{警告},第二页\textbf<2>{加粗}.

    
\end{frame}
\section{Two}
\begin{frame}{Two Columns}
    \begin{columns}
        \column{0.5\textwidth}%占据一半页面文字长度
            第一栏
        \column{0.5\textwidth}
            第二栏
    \end{columns}
\end{frame}

\section{Three}
\begin{frame}{Two Columns}
    slide
\end{frame}

\begin{frame}{Two Columns}
    slide
\end{frame}
\end{document}
